\documentclass[a4paper,10pt]{article}

\usepackage[margin = 0.5in, tmargin = 0.25in, bmargin = 0.25in]{geometry}
\usepackage[utf8]{inputenc}
\usepackage[english, russian]{babel}
\usepackage{hyperref}
\usepackage{fontawesome}
\usepackage{graphicx}
\usepackage[dvipsnames, table]{xcolor}
\usepackage{tabularx}
\usepackage{titlesec}
\usepackage{changepage}

\title{Резюме}
\date{\today}
\author{Дмитрий Першин}

\pagestyle{empty}
\setlength{\parskip}{3pt}
\setlength{\parindent}{0em}

\definecolor{dark}{RGB}{45, 120, 160}
\definecolor{middle}{RGB}{210, 220, 255}
\definecolor{light}{RGB}{240, 240, 255}

\titleformat{\subsection}{\Large\bfseries\color{dark}}{}{0em}{}[{\titlerule[1.5pt]}]
\titlespacing{\subsection}{0pt}{5pt}{5pt}

\newcommand{\alarm}[1]{{\textcolor{BrickRed}{\textbf{#1}}}}
\newcommand{\notice}[1]{{\textcolor{dark}{\textbf{#1}}}}


\begin{document}

\subsection*{Дмитрий Першин}

    \begin{adjustwidth}{10pt}{10pt}

        \setlength{\extrarowheight}{5pt}
        \begin{tabular}{ l l }
            \href{mailto:dapper91@mail.ru}{\faEnvelope~dapper91@mail.ru} &
            \href{https://www.linkedin.com/in/dapper91}{\faLinkedinSquare~www.linkedin.com/in/dapper91} \\
            \href{tel:+79126185850}{\faPhone~+7~912~618~5850} &
            \href{https://github.com/dapper91}{\faGithub~github.com/dapper91} \\

            \href{https://t.me/dapper91}{\faSend~@dapper91} &
            \href{https://dapper91.github.io/}{\faInfoCircle~dapper91.github.io} \\
            \href{https://www.google.com/maps/place/Gorod+Yekaterinburg,+Sverdlovsk+Oblast}{\faLocationArrow~Екатеринбург, Россия} & \\
        \end{tabular}
        \hfill
        \raisebox{-0.5\height}{\includegraphics[width=75pt]{info-qr.png}}

        Инженер-программист с опытом более 6 лет, участвовал в разработке проектов в области информационной безопасности, телекоммуникаций, облачных вычислений и сервисов, банковских сервисов. В поиске интересного проекта и профессиональной команды для решения амбициозных задач.

    \end{adjustwidth}

\subsection*{Образование}

    \begin{adjustwidth}{10pt}{10pt}

        \href{https://urfu.ru/en/}{Уральский Федеральный Университет}, \notice{Магистр информационных технологий}, радиотехнический факультет, \notice{диплом с отличием}

    \end{adjustwidth}


\subsection*{Опыт}

    \begin{adjustwidth}{10pt}{10pt}

        \notice{Инженер-программист в "Банк Точка"} / \textit{05.2019 – нас. время}

        \begin{itemize}

        \item Спроектировал, разработал с нуля и запустил в продакшн шлюз \href{https://sbp.nspk.ru/}{Системы быстрых платежей}. СБП - это сервис, который позволяет физическим лицам мгновенно совершать межбанковские переводы по номеру телефона, и производить оплату товаров по QR-коду. Разработанный сервис имеет высокую надежность, обслуживает клиентов 24/7, имеет гибкую архитектуру, что позволяет оперативно добавлять новые протоколы обмена.

        \item Спроектировал, разработал и сконфигурировал процесс CI/CD, включая юнит- и интеграционное тестирование продукта, анализ кода, сборку и деплой приложения. Это позволило увеличить качество кода, уменьшило число ошибок и сократило время релиза новой версии приложения до нескольких минут.

        \item Принял участие в разработке web-инструментов для финансовой аналитики. Инструменты предоставили пользователям множество визуальной информации в виде графиков, таблиц, основываясь на статистическом анализе, алгоритмах классификации и кластеризации, что позволило клиентам оптимизировать свои расходы. Сервис получил множество положительных отзывов от клиентов, большинство из которых, после окончания trial-периода, перешли на платную подписку.

        \end{itemize}


        \notice{Инженер-программист в "Яндекс"} / \textit{11.2018 – 05.2019}

        \begin{itemize}

        \item Принял участие в разработке распределенного высокодоступного высоконагруженного планировщика задач, который позволил более эффективно использовать ресурсы вычислительного кластера, и сократить затраты компании. Сервис хранил и обрабатывал более петабайта пользовательских данных, расположенных в нескольких датацентрах, обеспечивая доступность 24/7.

        \item Принял участие в разработке и развитии сервисного API и SDK, обеспечивающего возможность описания сложных вычислительных пайплайнов на нескольких тысячах серверов в нескольких датацентрах.

        \end{itemize}


        \notice{Инженер-программист в "Уральский центр систем безопасности"} / \textit{08.2016 – 11.2018}

        \begin{itemize}

        \item Принял участие в разработке программного обеспечения для анализа безопасности промышленных объектов в сети предприятий, предотвращения атак вредоносного ПО, анализа сетевого трафика. Продукт получил возможность обнаружения кибератак на сеть предприятия, оказания помощи в расследовании инцидентов на критически важных объектах производства.

        \item Переписал модули безопасности с многопоточной на асинхронную модель выполнения. Данная доработка увеличила производительность кода, и сделала возможным мониторинг большего числа сетевых объектов без обновления оборудования.

        \item Разработал модуль, использующий SCAP (security content automation protocol) стандарт для анализа защищенности операционных систем, используя открытую базу уязвимостей. Результатом работы стало сокращение времени обнаружения 0-day уязвимостей.

        \end{itemize}

    \end{adjustwidth}

\subsection*{Профессиональные навыки}

    \begin{adjustwidth}{10pt}{10pt}

        \begin{center}

        \begin{tabular}{
            >{\columncolor{light}}m{5em}
            >{\columncolor{white}}m{7em}
            >{\columncolor{light}}m{7em}
            >{\columncolor{white}}m{3em}
            >{\columncolor{light}}m{6em}
            >{\columncolor{white}}m{12em}
        }

        \rowcolor{middle}
        Языки & БД     & Фреймворки   & VCS & Другое     &  \\
        Python    & PostgreSQL    & asyncio      & Git & Linux      & SQL \\
        C/C++     & MongoDB       & aiohttp      &     & Docker     & BASH \\
        Rust      & Redis         & flask        &     & Kubernetes & Machine Learning Basics \\
                  & Elasticsearch & scikit-learn &     & RabbitMQ   & Distributed Systems \\
                  &               &              &     & Kafka      & CI/CD \\

        \end{tabular}

        \end{center}

    \end{adjustwidth}

\end{document}